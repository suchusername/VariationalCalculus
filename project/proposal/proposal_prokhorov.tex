% !TEX encoding = UTF-8 Unicode
\documentclass[14pt,a4paper]{article}

\usepackage[margin=2cm,includefoot]{geometry}
\usepackage[utf8]{inputenc}
\usepackage[russian]{babel}

\usepackage{lipsum} % generate filler text
\usepackage{hyperref} % create internal and external hyperlinks
\usepackage{amssymb} % extra math symbols
\usepackage{amsmath} % math environments
\usepackage{amsthm} % theorem and proofs environment
\usepackage{latexsym} % some special symbols
\usepackage{dsfont} % \mathbb{} symbols
\usepackage{mdframed} % for theorem boxes
\usepackage[linesnumbered]{algorithm2e} 
\usepackage{mathtools}
\usepackage{centernot}
\usepackage{xcolor}
\usepackage[shortlabels]{enumitem}
\usepackage{dirtytalk} % quotation marks
\usepackage{centernot} % \centernot

\usepackage{pifont} % for checkmark and cross
\newcommand{\cmark}{\ding{51}}%
\newcommand{\xmark}{\ding{55}}%

\usepackage{tikz} 
\usepackage{tikz-qtree}

\usepackage{graphicx}
\usepackage{float}
\graphicspath{ {./} }

\usepackage{tabularx}
\usepackage{makecell}
\usepackage{multicol}

\DeclarePairedDelimiter\ceil{\lceil}{\rceil}
\DeclarePairedDelimiter\floor{\lfloor}{\rfloor}

\def\changemargin#1#2{\list{}{\rightmargin#2\leftmargin#1}\item[]}
\let\endchangemargin=\endlist 

\newcommand\tab[1][0.7cm]{\hspace*{#1}}

\newcommand{\hm}[1]{#1\nobreak\discretionary{}{\hbox{\ensuremath{#1}}}{}}

%\renewcommand*{\qed}{\hfill\ensuremath{\blacksquare}}%

\usepackage{fancyhdr}
\pagestyle{fancy}
\fancyhead{}
\fancyfoot{}
\fancyfoot[R]{ \thepage\ }
\renewcommand{\headrulewidth}{0pt}
\renewcommand{\footrulewidth}{1pt}

\newmdenv[
  topline=false,
  bottomline=false,
  rightline=false,
  linewidth=1.5pt,
  skipabove=2.5pt,
  skipbelow=\topsep
]{theoremline}


%\let\oldproof\proof
%\renewcommand{\proof}{\color{gray}\oldproof}

%\begin{figure}[H]
%\centering
%\includegraphics[scale=0.5]{img2.png}
%\end{figure}

%\pushQED{\qed}
%\[ A + B = C \qedhere\]
%\popQED

%\begin{equation*}\begin{split}
%\end{split}\end{equation*}

%\begin{proof}[\color{gray}Доказательство.] $ $ {\small \color{gray}\\
%A + B = C \qedhere
%} \end{proof}

%\begin{center}\boxed{ A + B = C }\end{center}

%\\[-1em]

%\begin{equation*} A + B = C \tag{$\ast$}\end{equation*}

\begin{document}

	\begin{center} 
	\huge{\bfseries Нейросетевые методы решения задач вариационного исчисления} \\
	[0.5cm]
	\begin{tabular}{c   c} Прохоров Юрий  { $\ \ \ \ $   } &  prokhoryurij@gmail.com \end{tabular}
	{}\\[1em]
	\textsc{\Large Project proposal}\\[1em]
	\end{center}

\noindent В предлагаемой работе будут рассмотрены некоторые алгоритмы решения задач вариационного исчисления, основанные на \say{животных} принципах, а именно Whale Optimization Algorithm \cite{whale}, Gray Wolf Optimizer \cite{graywolf}, Moth Flame Optimization \cite{moth} и, возможно, некоторые другие. Будет проведено их общее описание, сравнение точности и скорости работы, а также будет предложена их модификация с использованием нейронных сетей. Также будет приведено интересное приложение данных алгоритмов для поиска путей.

\section{Постановка задачи}
В простейшем случае дана некоторая известная функция $L(t, x, \dot{x})$ и требуется решить задачу вариационного исчисления: минимизировать функционал
\begin{equation}J[x(t)] = \int\limits_a^b L\big(t, x(t), \dot{x}(t)\big) \, dt \longrightarrow \min \end{equation}
Иногда добавляются граничные условия.\\\\
Одной из основных идей является поиск решения в виде разложения по некоторому функциональному базису
\begin{equation}
x(t) = \sum_{k=1}^T c_k \mathbf{X}_k(t),
\end{equation}
где функции базиса выбираются из некоторых практических соображений (например, можно строить в виде сплайнов, как описано в \cite{nnarticle}).\\\\
Другой идеей является использование алгоритмов нулевого порядка. Например, метод Whale Optimization Algorithm \cite{whale} похож на генетический алгоритм и основан на создании популяции \say{китов}, которая будет последовательно \say{эволюционировать} и постепенно сходиться к решению.\\\\
Моя идея состоит в обобщении этих методов и использовании нейронных сетей для возможного улучшение качества их работы. Частично такое обобщение было проведено, например, в \cite{dissertation}.

\section{Планируемые результаты}

В качестве результата проекта я планирую получить нейросетевой алгоритм, который на вход будет принимать задачу вариационного исчисления, а на выходе будет давать ее численное решение.\\\\
Строгих теоретических обоснований к такому алгоритму, скорее всего, не будет, потому что многие алгоритмы нулевого порядка основаны на идеях, которые часто бывает сложно строго математически сформулировать.

\section{Литературный обзор}

Основная идея в разложении решения вариационной задачи по функциональному базису и математические методы численного решения описаны в \cite{appliedcalc} и \cite{chebyshev}. В работах \cite{nnarticle}, \cite{dissertation} рассказано, как это идею можно перенести на нейронные сети.\\\\
Упомянутые мною \say{животные} алгоритмы описаны в следующих работах: Whale optimization algorithm --- \cite{whale}, Gray wolf optimizer --- \cite{graywolf}, Moth flame optimization --- \cite{moth}. Также в работе \cite{whale2} приведено интересное улучшение \say{китового} алгоритма из \cite{whale}.\\\\
Также в статьях \cite{fastalgorithm} и \cite{splines} и приведены другие эффективные численные алгоритмы, которые я, возможно, также рассмотрю в своей работе.

\section{Метрики качества}

В качестве метрики качества предлагается следующая метрика. Рассмотрим некоторое количество $N$ задач вариационного исчисления, для которых возможно найти точное решение из дифференциального уравнения Эйлера-Лагранжа. Затем для каждой ($k$-ой) из этих задач рассмотрим оценку $L_1$-нормы между численным решением $x^k_{\text{num}}(t)$ и точным решением $x^k_\ast(t)$:
$$\big\lVert x_k^{\text{num}} - x_k^\ast \big\rVert \overset{def}{=} \frac{1}{N} \sum_{i=1}^{M_k} \big\lvert  x_k^{\text{num}}(t_i) - x_k^\ast(t_i) \big\rvert,$$
где $\{t_i\}$ --- некоторый набор точек. Тогда в качестве метрики качества можно взять метрику:
$$L(\mathbf{A}) = \sum_{k=1}^N \big\lVert x_k^{\text{num}} - x_k^\ast \big\rVert,$$
где $\mathbf{A}$ --- алгоритм, строящий численные решения.\\\\
Такая метрика предложена в нескольких работах, например, в \cite{whale}.

\section{Примерный план}

\begin{enumerate}
\item Сейчас я читаю про реализацию алгоритмов вариационного исчисления с помощью нейронных сетей в \cite{nnarticle} и \cite{dissertation}.
\item Скоро я начну разобраться в принципах работы \say{животных} алгоритмов \cite{whale}, \cite{graywolf} и \cite{moth}. 
\item Потом я попытаюсь скрестить эти методы, используя некоторые идеи из \cite{whale2}.
\item Потом я получу сравнение всех изученных мною методов.
\item Если успею, то приготовлю графическую анимацию этого метода на примере поиска наискорейшего пути.
\end{enumerate}

\begin{thebibliography}{100}

\bibitem{nnarticle} 
Andrew J. Meade Jr., Hans C. Sonneborn.
\textit{Numerical solution of a calculus of variations problem using the feedforward neural network architecture}. 
Department of Mechanical Engineering and Materials Science, William Marsh Rice University, Houston, TX, USA, 1996.

\bibitem{dissertation} 
Roberto Lopez Gonzalez.
\textit{Neural Networks for Variational Problems in Engineering}. 
Artificial Intelligence Department of Computer Languages and Systems, Technical University of Catalonia, 2008.

\bibitem{whale} 
Seyed Hamed Hashemi Mehne, Seyedali Mirjalili.
\textit{A direct method for solving calculus of variations problems using the whale optimization algorithm}. 
Institute for Integrated and Intelligent Systems, Grifth University, Nathan Campus, Brisbane, Australia, 2019

\bibitem{whale2} 
Seyed Mostafa Bozorgi, Samaneh Yazdani.
\textit{IWOA: An improved whale optimization algorithm for optimization problems}. 
Department of Computer Engineering, North Tehran Branch, Islamic Azad University, Tehran, Iran, 2019

\bibitem{graywolf} 
Mirjalili S, Mirjalili SM, Lewis A.
\textit{Grey wolf optimizer.}
Adv Eng Softw 69:46–61, 2014

\bibitem{moth} 
Mohammad Shehab, Laith Abualigah, Husam Al Hamad, Hamzeh Alabool, Mohammad Alshinwan, Ahmad M. Khasawneh.
\textit{Moth–flame optimization algorithm: variants and applications.}
Faculty of Computer Sciences and Informatics, Amman Arab
University, Amman, Jordan, 2019

\bibitem{appliedcalc} 
Komzsik L.
\textit{Applied calculus of variations for engineers.}
CRC Press, Boca Raton, 2009

\bibitem{fastalgorithm} 
A. R. Nazemi, S. Hesam, A. Haghbin.
\textit{A fast numerical method for solving calculus of variation problems}. 
Department of Mathematics, School of Mathematical Sciences, Shahrood University of Technology, 2013.

\bibitem{splines} 
M. Zarebnia, M. Birjandi.
\textit{The Numerical Solution of Problems in Calculus of Variation Using B-Spline Collocation Method}. 
Department of Mathematics, University of Mohaghegh Ardabili, Iran, 2012

\bibitem{chebyshev}
Hadi Rostamzadeh, Mohammad Lotfi, Keivan Mostoufi.
\textit{Application of Chebyshev Finite Difference Method (ChFDM) in Calculus of Variation.}
Niroo Research Institute, Tehran, Iran, 2016

\end{thebibliography}


\end{document}



